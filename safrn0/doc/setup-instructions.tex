\documentclass[12pt]{article}
\usepackage{listings}

\usepackage{amsmath, amsfonts}
\usepackage{hyperref}

\begin{document}
\lstset{language=sh}
\lstset{language=sql}
\lstset{language=xml}


\title{SAFRN setup}


\section{Install MySQL}

This is all cribbed from steps 1 and 2 in \url{https://dev.mysql.com/doc/mysql-apt-repo-quick-guide/en}.


\begin{lstlisting}[language=sh]
  sudo apt-get install mysql-server
  sudo mysql_secure_installation
\end{lstlisting}

\subsection{Annoying problem with sudo}

I followed the steps outlined in
\url{https://stackoverflow.com/questions/39281594/error-1698-28000-access-denied-for-user-rootlocalhost}
so that I didn't need to keep typing $sudo$ with every MySql
execution.

\begin{enumerate}
\item Start a MySql shell:
\begin{lstlisting}[language=sh]
sudo mysql -u root -p
\end{lstlisting}

\item Run these sql commands to change the plugin for $root$.
\begin{lstlisting}[language=sql]
USE mysql;
UPDATE user SET plugin='mysql_native_password' WHERE User='root';
FLUSH PRIVILEGES;
exit;
\end{lstlisting}

\item Restart the service
\begin{lstlisting}[language=sh]
service mysql restart
\end{lstlisting}
\end{enumerate}

After that you can run MySQL without $sudo$.

\section{loading sample data}

\begin{enumerate}




\item Download the sample data from Google. We will denote the
  downloaded directory by the environment variable $SAFRN_DATA$.
\item Prepare the neccessary sql scripts:
  \begin{lstlisting}[language=sh]

bash ${SAFRN_DATA}/create_sql_group.bash 1 ${SAFRN_DATA}/files/1K/MPC_Test5_1K_School1.csv \
> ${SAFRN_DATA}/files/group1/safrngroup1.sql

bash ${SAFRN_DATA}/create_sql_group.bash 2 ${SAFRN_DATA}/files/1K/MPC_Test5_1K_School2.csv \
> ${SAFRN_DATA}/files/group1/safrngroup2.sql

bash ${SAFRN_DATA}/create_sql_group.bash 3 ${SAFRN_DATA}/files/1K/MPC_Test5_1K_School3.csv \
> ${SAFRN_DATA}/files/group1/safrngroup3.sql

bash ${SAFRN_DATA}/create_sql_group.bash 4 ${SAFRN_DATA}/files/1K/MPC_Test5_1K_School4.csv \
> ${SAFRN_DATA}/files/group1/safrngroup4.sql

bash ${SAFRN_DATA}/create_sql_income.bash ${SAFRN_DATA}/files/1K/MPC_Test5_1K_income.csv \
> ${SAFRN_DATA}/files/income/safrnincome.sql

bash ${SAFRN_DATA}/create_sql_loans.bash ${SAFRN_DATA}/files/1K/MPC_Test5_1K_loans.csv \
> ${SAFRN_DATA}/files/loans/safrnloans.sql

\end{lstlisting}

\item Run those scripts:
  \begin{lstlisting}[language=sh]
    mysql -u root < $SAFRN_DATA/files/group1/safrngroup1.sql
    mysql -u root < $SAFRN_DATA/files/group2/safrngroup2.sql
    mysql -u root < $SAFRN_DATA/files/group3/safrngroup3.sql
    mysql -u root < $SAFRN_DATA/files/group4/safrngroup4.sql
    mysql -u root < $SAFRN_DATA/files/income/safrnincome.sql
    mysql -u root < $SAFRN_DATA/files/loans/safrnloans.sql
  \end{lstlisting}

\end{enumerate}
Running these scripts will create the tables $SAFRN2Group1.GroupTbl$,
$SAFRN2Group2.GroupTbl$, $SAFRN2Group3.GroupTbl$,
$SAFRN2Group4.GroupTbl$, $SAFRN2Income.IncomeTbl$,
$SAFRN2Loans.LoansTbl$, and insert the data from the csvs into those
tables.

\section{Front End}

The front-end code can be found here
\url{https://github.com/ICPSR/safrn}

To build it you will need to install the JVM build tool Maven
\url{https://maven.apache.org/install.html}.

You can build the artifact by using the war (Web Application Resource)
plugin \url{https://maven.apache.org/plugins/maven-war-plugin}

\begin{lstlisting}[language=sh]
mvn clean war:war
\end{lstlisting}

This will create a file $./target/safrn.war$. Next we use the maven
plugin tomcat \url{https://tomcat.apache.org/maven-plugin.html} to run
the web app locally. Add the necessary plugin dependency to the
$./pom.xml$ file:
\begin{lstlisting}[language=xml]

  <build>
...
  <plugins>
      <plugin>
        <groupId>org.apache.tomcat.maven</groupId>
        <artifactId>tomcat7-maven-plugin</artifactId>
        <version>2.2</version>
        <configuration>
          <port>8080</port>   //Configure port number
          <path>/safrn-demo</path>   //Configure application root URL
        </configuration>
      </plugin>
    </plugins>
...
</build>

\end{lstlisting}

Then start the webapp with:

\begin{lstlisting}[language=sh]
mvn clean tomcat7:run
\end{lstlisting}

You should be able to see the app running at
\url{http://localhost:8080/safrn-demo/} (Note that the port 8080 and
the path safrn-demo are both configurable).


\end{document}
